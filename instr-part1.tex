\section*{Project I: Topic ID and Overview (due Sept 17)}

%This is just a placeholder for now

The first step in your semester-long research proposal development is to select a topic area in the mechanics of soft materials that's sufficiently interesting to you. 
It may be helpful to think of the proposal-writing process as the following. 
You want to study something that you are especially interested in, but you don't yet have the resources that you need to pursue this fully. 
Your job is (eventually) to communicate what it is you want to study, why it's worthwhile to be studied, and enumerate all of the reasons it's in some benefactor's interest to provide you the support that you need.
The particular benefactor we will leverage is the National Science Foundation, which cares about making fundamental ``vertical'' advances in fields (known as their \textit{Intellectual Merit} criterion) and having their funded projects improve society (known as the \textit{Broader Impact} to society criterion). 

Aim for approximately 500 words of total text, such as to reflect the important three Cs: \textbf{\textit{clear}}, \textbf{\textit{concise}}, and \textbf{\textit{compelling}}.
Your submission should be structured in three sections as separated below, and should address the following points: 

\begin{enumerate}
\item \textbf{Statement of Research Interest (why you personally want to study the subject)}
\begin{itemize}
\item Describe an area or phenomenon in the mechanics of soft materials that you find compelling.
\item What motivates your interest and pursuit of this subject (e.g., your current or developing expertise, research interests, or otherwise)? \textit{Note: This may be more personal or anecdotal and is for my own understanding of your topic selection!}

The unique dependency of soft materials’ response on strain rate is what makes this class of materials stand out from their stiffer counterparts. In particular, the balance between energy dissipation and energy return under high strain rates is both scientifically fascinating and technologically vital. While soft materials must be sufficiently stiff to provide stability, they also excel at shock absorption in many applications. Understanding the mechanisms that govern this trade-off offers both fundamental insight and guidance for materials design across multiple length scales.

My personal motivation stems from my long-standing interest in basketball. For years, I have been curious about how athletic shoes can provide both stable support for joints and explosive responsiveness while simultaneously reducing long-term impact on the body to prolong athletes’ careers. This curiosity motivates me to focus on ethylene-vinyl acetate (EVA) foam, the most common cushioning material in performance footwear. EVA serves as an ideal platform to study how microstructural mechanics under high strain rates can be linked to macroscopic constitutive response, with direct implications for human performance and health.
\end{itemize}
\item \textbf{Intellectual Merit (why it is objectively worth delving deeper)}
\begin{itemize}
\item Describe, to someone with expertise in mechanics but perhaps not your system of interest, the core scientific principles underpinning (or perhaps, enabling development in) your topic of interest.
\item Given the course syllabus, how will particular material we will cover this semester relate to what you propose? What background information do you need to do not just a good, but great, job in proposing something interesting?

A central scientific question in this study is: How do microstructural features of EVA foam govern its macroscopic constitutive response under dynamic loading? While industry often relies on empirical characterization of bulk properties for design, advancing our understanding requires bridging scales: from the viscoelastic behavior of polymer chains, to the cellular architecture of foams, and eventually to the stress-strain response at the continuum level.

Addressing this problem demands a dual approach. First, experimental investigation of rate-dependent constitutive laws can reveal the viscoelastic/viscoplastic behavior at high strain rates. Second, microstructural studies, such as foam cell morphology, can uncover how local buckling, collapse, and recovery contribute to global energy dissipation and energy return. Together, these perspectives enable not just better material models, but also insights into how processing and formulation choices ultimately determine performance.

The topics in this course provide the theoretical foundation for such a study. Linear viscoelasticity is central to building constitutive models, while nonlinear elasticity and fracture mechanics will help interpret high-strain-rate behavior and potential failure modes. Micro-mechanical models discussed in class will serve as theoretical anchors for interpreting lab results. To do this work well, I will also gather upstream knowledge of EVA’s formulation and manufacturing processes, and compare with other soft material systems used in sports and protective equipment (e.g., helmet).
\end{itemize}
\item \textbf{Broader Impact (who, or what, does studying this area benefit?)}
\begin{itemize}
\item How might advances you envision in this area be impactful beyond your own interest? 
\item What does a ``winning scenario'' in this area look like? Briefly describe who might benefit (e.g., particular industries, health/science sectors, the public) and how that could plausibly happen.

Although motivated by athletic footwear, this research extends far beyond sports. Understanding how EVA and similar foams dissipate energy under dynamic loading could improve rehabilitation devices for patients recovering from joint injuries, where stability and impact mitigation are critical. In the transportation sector, foams are widely used in automotive and aircraft seating, interior panels, and protective padding, where improved cushioning could enhance safety and comfort.

A “winning scenario” would mean developing a predictive framework that links microstructure, processing, and performance. This would allow designers in sports, healthcare, and transportation to move beyond trial-and-error testing and toward rational, science-driven material selection. Society would benefit through safer, longer-lasting consumer products, reduced injury risks, and more efficient use of polymeric materials in diverse applications.
\end{itemize}
\end{enumerate}

\emph{A strong submission will clearly illustrate your personal goals with this project, and show how your interest could manifest as advances in the broader field and society.}





