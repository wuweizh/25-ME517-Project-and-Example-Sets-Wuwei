\section*{Project II: Literature Review (due Oct 3)}

The second step in your semester-long research proposal development is to contextualize your problem within the current field of your choice, demonstrate your understanding of the state-of-the-art, and identify something we don't yet understand but need to---this is the gap your (hypothetical) proposed project would seek to fill.

You'll execute this portion of the project as an outline. 
This outline does not need to be long! 
It does, however, need to be very clear, as you'll be expanding on it later. 
The sections should be the following:

\renewcommand{\outlinei}{enumerate}
\renewcommand{\outlineii}{itemize}
\begin{outline}
    \1 \textbf{Introductory context}
        \2 Add one or two bullet points briefly framing the history of your topic and its significance. 
        \2 Citations here are important---there should be several well-placed citations in each of these bullet points, and good review papers are especially helpful to lean on for context. 

        \2 Polymeric foams such as ethylene-vinyl acetate (EVA) have been widely used for energy absorption and cushioning since their introduction into the footwear industry in the 1970s, owing to their lightweight nature and tunable mechanical response \cite{tomin2021polymer, lunchev2022eva, sun2020footwear}. 
        \2 Recent reviews emphasize the growing role of advanced polymer foams in sports applications, highlighting how microstructural design can drive performance and injury-prevention innovations \cite{tomin2021polymer, desouza2023polymeric}. 
        \2 At the fundamental level, the strain-rate sensitivity of cellular polymeric foams arises from a combination of polymer chain viscoelasticity and cell-wall buckling/collapse, linking molecular dynamics to bulk mechanical response \cite{desouza2023polymeric}.  

    \1 \textbf{The state of the field}
        \2 In one or two bullet points, explain how things are done in the area of your project at the present moment. 
        \2 This can include any of experimentation, computational methods, and/or theory.
         \2 Experimentally, dynamic mechanical analysis (DMA), drop-weight impact tests, and high-speed compression experiments are widely used to characterize EVA foams, capturing their stress--strain behavior across strain rates \cite{chen2023eva}. High-strain-rate methods combined with digital image correlation have become standard tools for probing impact response in polymeric foams \cite{liu2014impact, koohbor2016dynamic}.
        \2 Computationally, finite element methods, including hyper-viscoelastic and crushable foam models \cite{fazekas2018hyper}, are commonly used to approximate energy absorption and resilience \cite{li2019eva}. More recent adaptive FE approaches allow the effective mechanical properties of porous elastomeric materials to be reflected with greater fidelity \cite{nazari2022adaptive}, though EVA-specific microstructural inputs remain limited. 
    \1 \textbf{The Big Gap}
        \2 In one or two sentences, what is it that we don't know, and why isn't it solved? 

        \2 Despite decades of use, the fundamental link between EVA’s microstructure (e.g., cell morphology, polymer phase interactions) and its macroscopic rate-dependent constitutive behavior remains poorly understood. 
        \2 Current models either oversimplify or are too computationally intensive, leaving a big gap in predictive, experimentally validated frameworks that can both explain and guide foam design at multiple scales.

\end{outline}

\textit{A strong literature review outline will succinctly illustrate the essential context of the problem, the current best knowledge in the area, and the critical gap in knowledge restricting further advances or implementation, and should cite approximately 10-15 references with a \LaTeX ~bibliography.}


%This is just a placeholder for now


