\setcounter{section}{1} 
\section*{Solutions to Examples II: Kinetics, Constitutive Laws, and Viscoelasticity I}
\label{soln-PS2}

\medskip
\subsection*{2--1. \textbf{Balance of mass} [4 pts].} 
We can start with the general form of the conservation of mass equation:
\begin{equation}
\label{eq:continuity}
    \frac{\partial \rho}{\partial t} + \bm{\nabla}_{\bm{x}} \rho \cdot\bm{v} + \rho \bm{\nabla}_{\bm{x}} \cdot \bm{v} = 0.
\end{equation}
    Now we want to incorporate the gradient in spherical coordinates, which will give us a few terms for each of the second and third terms here. 
    The gradient of a scalar function in spherical coordinates is:
\begin{equation*}
    \bm{\nabla}_{\bm{x}} \rho = \frac{\partial \rho}{\partial r} \bm{e}_r + \frac{1}{r} \frac{\partial \rho}{\partial \theta} \bm{e}_\theta  + \frac{1}{r\sin\theta}\frac{\partial \rho}{\partial \phi} \bm{e}_\phi  
\end{equation*}
The divergence of a vector function $\bm{v}$ is:
\begin{equation*}
    \bm{\nabla}_{\bm{x}} \cdot \bm{v} = \frac{\partial v_r}{\partial r} + 2\frac{v_r}{r}+ \frac{1}{r}\frac{\partial v_\theta}{\partial \theta} + \frac{1}{r \sin\theta}\frac{\partial v_\phi}{\partial \phi} + \frac{v_\theta}{r}\cot\theta.
\end{equation*}
Now, since $\bm{v} = v_r \bm{e}_r$, we can drop all of the partial derivatives with respect to $\theta$ and $\phi$, which leaves the divergence of velocity as
\begin{equation*}
    \bm{\nabla}_{\bm{x}} \cdot \bm{v} = \frac{\partial v_r}{\partial r} + 2\frac{v_r}{r}.
\end{equation*}
Incorporating incompressibility means that the density $\rho$ cannot be a function of $\bm{x}$ or $t$, so the time derivatives and gradient of $\rho$ corresponding to the first two terms, respectively, in Eq. \ref{eq:continuity} must be zero. 
This leaves
\begin{equation*}
    \frac{\partial v_r}{\partial r} + 2\frac{v_r}{r} = 0,
\end{equation*}
which we can separate and integrate from the bubble wall to a position with a corresponding velocity as follows:
\begin{align*}
    \frac{\partial v_r}{\partial r} &= -2\frac{v_r}{r}\\
    \int_{\dot{R}}^{v_r} \frac{1}{\tilde{v}_r} d\tilde{v}_r &= -2\int_R^r \frac{1}{\tilde{r}} d\tilde{r}\\
    \ln\left(\frac{v_r}{\dot{R}}\right) &= -2 \ln\left(\frac{r}{R}\right)\\
    v_r &= \frac{R^2}{r^2}\dot{R} ~.//
\end{align*}


\medskip
\subsection*{2--2. \textbf{Balance of momenta} [4 pts].}

To start, let's write out the balance of linear momentum.
\begin{equation*}
    \int_{\partial\Omega} \bm{t}~dA + \int_{\Omega} \bm{b} ~dV = \int_{\Omega} \rho \bm{a} ~dV.
\end{equation*}
From here, we can calculate each of the individual integrals on the left hand side. 
The first will benefit from the divergence theorem, which can turn it from a surface integral to a simple volumetric one of the divergence of the stress, which is:
\begin{equation*}
    \bm{t} = \bm{n} \cdot \bm{\sigma} \Rightarrow \bm{\sigma} = -\rho_w g x_3 \bn{I}.
\end{equation*}
The stress can then be used in the volume integral:
\begin{equation*}
    \int_{\partial\Omega} \bm{t}~dA  = \int_{\Omega} \bm{\nabla}_{\bm{x}} \cdot \bm{\sigma} ~dV.
\end{equation*}
As the stress is just a function of $x_3$, the dot product with the gradient operator simplifies to a derivative with respect to $x_3$ only, and the integrand evaluates to:
\begin{equation*}
\bm{\nabla}_{\bm{x}} \cdot \bm{\sigma} = \frac{\partial }{\partial x_3} \left( -\rho_w g x_3 \right) = -\rho_w g.
\end{equation*}
Integrating this constant stress divergence over the volume gives us a total integrated traction of $-4\pi\rho_w g R^3/3$. 
The body force integral is a bit more straightforward as it's just the gravitational force, but we should include the gradient in $x_2$, which is $\rho(\bm{x}) = \rho_w \left(1 + \frac{1}{2}\frac{x_2}{R}\right)$. 
We can be clever with our choice of the primary axis of our sphere to coincide with the direction of $\bm{e}_2$, which makes $x_2 = r\cos\phi$ and $x_1$ and $x_3$ some combination of $\pm r\cos\theta\sin\phi$ and $\pm r\sin\theta\sin\phi$. 
If we want both to be positive, we take $x_1 = r\cos\theta\sin\phi$ and $x_3 = r\sin\theta\sin\phi$. 
Integrating the body force in spherical coordinates (more important for later than now) then becomes
\begin{align*}
    \int_{\Omega} \bm{b} ~dV &= \int_0^\pi \int_0^R \int_0^{2\pi} \rho_w \left(1+\frac{1}{2}\cos\phi\right) g~r^2 \sin\phi ~d\theta dr d\phi\\
    &= 2\pi\rho_w g \int_0^\pi \int_0^R  \left(1+\frac{1}{2}\cos\phi\right)~r^2 \sin\phi ~ dr d\phi\\
    &= 2\pi \frac{R^3}{3}\rho_w g \left.\left(-\cos\theta -\frac{\cos^2\theta}{4}  \right)\right|_0^{\pi}\\
    &= \frac{4\pi R^3}{3}\rho_w g
\end{align*}
The body force is the weight due to gravity, which perfectly offsets the traction on the region occupied by the sphere supplied by the displaced water. 
Thus, this is naturally in equilibrium!
Note though, that this is ``for now''. 
If the sphere were to rotate we'd have to change the direction of the gradient, replacing the $\cos\theta/2$ with a different Cartesian direction's representation in spherical. 
More on this in a bit!

Now we want to consider the integrals of the moment terms. 
The integral for the traction becomes:
\begin{equation*}
    \int_{\partial \Omega} \bm{x} \times \bm{t}~dA  = \int_{\partial\Omega} \bm{x} \times (\nabla_{\bm{x}} \cdot \bm{\sigma})~d
    A. 
\end{equation*}
We can apply the divergence theorem and write this in index notation now, and manipulate this further:
\begin{align*}
   & \int_{ \Omega} \epsilon_{ijk} \frac{\partial}{\partial x_m} (x_j \sigma_{mk}) \bm{e}_i ~dV\\
   =& \int_{ \Omega} -\rho_w g \epsilon_{ijk} \frac{\partial}{\partial x_m} (x_j x_3 \delta_{mk}) \bm{e}_i ~dV\\
   =& -\rho_w g \int_{ \Omega}  \epsilon_{ijk} \frac{\partial}{\partial x_k} (x_j x_3) \bm{e}_i ~dV\\
   =& -\rho_w g \int_{ \Omega}  \epsilon_{ij3}x_j \bm{e}_i ~dV\\
    =& -\rho_w g \int_{ \Omega}  \epsilon_{ijk} (\cancelto{0, \textrm{ since } \epsilon_{ijj} = 0}{x_3 \delta_{jk}} + x_j \delta_{3k}) \bm{e}_i ~dV\\
    =& -\rho_w g \int_{ \Omega}  \epsilon_{ij3}x_j \bm{e}_i ~dV\\
    =& -\rho_w g \int_{ \Omega}  (x_2 \bm{e}_1 - x_1 \bm{e}_2) ~dV.
\end{align*}

Now, let's deal with the body force term:
\begin{equation*}
    \int_{ \Omega} \bm{x} \times \bm{b}~dV  = \int_{ \Omega} \bm{x} \times \rho(\bm{x}) g \bm{e}_3~dV.
\end{equation*}
The density function doesn't affect the cross product, which, in index notation, is:
\begin{align*}
    \int_\Omega \rho(\bm{x}) g\epsilon_{ij3} x_j \bm{e}_3 \bm{e}_i dV\\
    =\int_\Omega \rho(\bm{x}) g(x_2 \bm{e}_1 - x_1 \bm{e}_2) dV.
\end{align*}
The sum of the two terms is (which would be zero if we're in rotational equilibrium):
\begin{align*}
    \int_\Omega (\rho(\bm{x}) g - \rho_w g)(x_2 \bm{e}_1 - x_1 \bm{e}_2) dV.
\end{align*}
Now we'll write everything in spherical coordinates as above,
\begin{align*}
   & \int_0^\pi \int_0^{2\pi} \int_0^R \left( \rho_w\left(1+\frac{1}{2}   \cos\phi\right)g-\rho_w g \right) r^2 \sin\phi \left(r\cos\phi \bm{e}_1 - r \cos\theta \sin\phi \bm{e}_2 \right) dr d\theta d\phi\\
   =&\frac{1}{2}\rho_w g \int_0^\pi \int_0^{2\pi} \int_0^R r^3 \cos\phi \sin\phi \left(\cos\phi \bm{e}_1 - \cos\theta \sin\phi \bm{e}_2 \right) dr d\theta d\phi\\
   =&\frac{1}{8}R^4 \rho_w g \int_0^\pi \int_0^{2\pi}  \left(\cos^2\phi \sin\phi \bm{e}_1 - \cos\theta \cos\phi \sin^2 \phi \bm{e}_2 \right)  d\theta d\phi\\
    =&\frac{\pi}{4}R^4 \rho_w g \int_0^\pi  \cos^2\phi \sin\phi \bm{e}_1  d\phi\\
    =&\frac{\pi}{4}R^4 \rho_w g \cdot -\frac{1}{3}\cos^3\phi\Big|_0^\pi \bm{e}_1 \\
    =& \frac{\pi}{6} \rho_w g R^4 \bm{e}_1.
\end{align*}
Note what this means---there's a non-zero moment! 
This moment arises because the horizontal gradient requires a moment to sustain. 
What would happen is a rotation such that the gradient would settle vertically, such that the bottom would be the densest and the top would be the least dense. 

\bigskip
\subsection*{2--3. \textbf{Viscoelastic data} [4 pts].} 

(a) For linear viscoelasticity, we require the material to respond linearly (i.e. stress proportional to strain) regardless of the time, $t$. 
How this slope changes is the function of time we care about, but the slope must be constant over some region of interest---in this case, something like $\varepsilon<1.5\times10^{-3}$. 
This is in contrast to e.g. quasi-linear viscoelasticity (QLV), where a material may have a non-linear stress-strain response but this functional form $f(\varepsilon)$ must not change based on time. 

(b) The strain data at 100 Pa and 250 Pa are:
\begin{table}[h]
    \caption{Strain values for different stress levels. All strains are $\times 10^{-3}$.}
    \centering
\begin{tabular}{c c c}
     & \textbf{$\bm{\sigma_1=}$ 100 Pa}  & \textbf{$\bm{\sigma_2=}$ 250 Pa}\\ \hline
      \textbf{Time} & \textbf{Strain}  & \textbf{Strain}\\ \hline
    2 & 0.45 & 1.20\\
    5 & 0.53 & 1.45\\
    10 & 0.68 & 1.80\\
    20 & 0.84 & 2.40\\
    40 & 1.00 & 3.00
\end{tabular}
\end{table}

The creep compliance $J_c(t)$ at any time will be $J_c(t) = \varepsilon(t)/\sigma_0$. 
Rewriting the tables as creep compliance, we get the following.
\begin{table}[H]
    \caption{Creep compliance $J_c(t)$ at stress levels of 100 Pa and 250 Pa.}
    \centering
\begin{tabular}{c c c}
    & \textbf{$\bm{\sigma_1=}$ 100 Pa}  & \textbf{$\bm{\sigma_2=}$ 250 Pa}\\ \hline
   \textbf{Time} & \textbf{Compliance}  & \textbf{Compliance}\\ \hline
    2 & 0.45 $\times 10^{-5}$ & 0.48 $\times 10^{-5}$\\
    5 & 0.53 $\times 10^{-5}$ & 0.58 $\times 10^{-5}$\\
    10 & 0.68 $\times 10^{-5}$ & 0.72 $\times 10^{-5}$\\
    20 & 0.84 $\times 10^{-5}$ & 0.96 $\times 10^{-5}$\\
    40 & 1.00 $\times 10^{-5}$ & 1.20 $\times 10^{-5}$
\end{tabular}
\end{table}
We now need to figure out the creep compliance function. 
The creep compliance ought to increase as a function of time as the material softens. 
We may not necessarily know the functional form of this response, but we expect a stiff baseline behavior and subsequent relaxation. 
We probably expect it to also plateau to some other value at long times (that we did not measure). 

Other than that, I'd suggest trying something with an exponential, e.g. fitting to $J_c(t) = J_\infty(1-\exp(-t/\tau))$, which is the Kelvin-Voigt material model.
 To do this, we just divide by $J_
\infty$ and take a log of the expression:
\begin{equation*}
     \ln \left(1-\frac{J_c(t)}{J_\infty}\right) = -t/\tau.
\end{equation*}
There'd be a considerable assumption using $J_\infty$ as the value at 40 seconds, but it may be usable for determining the relaxation regardless. 
The result is a relaxation time on the order of $\approx 4-10$, monotonically increasing. 
What we can take away from this is that the relaxation time actually continues to increase as the material is loaded; this suggests either a spectrum of relaxation, some other functional dependence, or that we're missing complexity by using this model.

\bigskip
\subsection*{2--4. \textbf{Impulsive stresses} [4 pts].}

(a) We start with the general form for strain as a function of stress in a viscoelastic material:
\begin{equation*}
    \varepsilon(t) = \int_{0^-}^tJ_c(t-\tau) \dot{\sigma}(\tau) d\tau.
\end{equation*}
This is okay for the first part of our problem once we substitute in our actual stress function:
\begin{equation*}
     \varepsilon(t) = \int_{0^-}^tJ_c(t-\tau)  A \psi(\tau) d\tau.
\end{equation*}
Then, we need to use a specific $J_c(t)$ for the second half. 
The Kelvin-Voigt function for $J_c(t)$ is
\begin{equation*}
    J_c(t) = \frac{1}{E} \left( 1 - e^{-Et/\eta}\right).
\end{equation*}
This will be most straightforward to approach in Laplace space.
\begin{equation*}
    J(t)*\dot{\sigma}(t) = \bar{J}(s)\cdot \bar{\dot{\sigma}}(s) =  s\bar{{\sigma}}(s)\bar{J}(s).
\end{equation*}

We just need some transforms now.
\begin{align*}
    \bar{\varepsilon}(s) &= \frac{1}{E} \left( \frac{1}{s} - \frac{1}{s+\frac{E}{\eta}}\right) \cdot s\cdot A\\
    &= \frac{A}{E}\left( 1 - \frac{s}{s+\frac{E}{\eta}} \right)\\
    \varepsilon(t) &= \frac{A}{\eta} e^{-Et/\eta}.
\end{align*}


(b) We follow up with a doublet stress, where $\sigma(t) = B\psi(t)$. 
\begin{align*}
    \bar{\varepsilon}(s) &= \frac{1}{E} \left( \frac{1}{s} - \frac{1}{s+\frac{E}{\eta}}\right) \cdot s\cdot Bs\\
    &= \frac{B}{E}\left( s - \frac{s^2}{s+\frac{E}{\eta}} \right)\\
    \varepsilon(t) &= \frac{B}{E}\delta(t) - \frac{BE}{\eta^2} e^{-Et/\eta}.
    %\varepsilon(t) &=\frac{B}{\eta}\psi(t) - \frac{B E}{\eta^2}\delta(t) +  \frac{BE^2}{\eta^3} e^{-Et/\eta}.
\end{align*}
Note that this will converge to $\varepsilon(t>0^+)=- \frac{BE}{\eta^2} e^{-Et/\eta}$ after the initial delta function. 


\bigskip
\subsection*{2--5. \textbf{Two-element models} [8 pts].}

(a+b) The differential constitutive law for a Kelvin-Voigt material is
\begin{equation*}
    \eta \dot{\varepsilon}(t) + E \varepsilon(t) = \sigma(t). 
\end{equation*}
If we take the strain to be $\varepsilon(t) = -\varepsilon_0 \mathcal{H}(t) - \varepsilon_0 \sin\omega t$, we just need to take a derivative of the strain function to get the strain rate: $\dot{\varepsilon}(t) = -\omega \varepsilon_0 \cos\omega t + \varepsilon_0\delta(t)$.
We can then get the stress,
\begin{align*}
    \sigma(t) &= \omega \eta (-\varepsilon_0\cos\omega t - \varepsilon_0 \delta(t)) + E(-\varepsilon_0 - \varepsilon_0 \mathcal{H}(t) \sin\omega t).
\end{align*}
From here, we can use an identity to combine the cosine and sine terms:
\begin{equation*}
     A \sin \omega t + B \cos \omega t= \sqrt{A^2 + B^2} \sin (\omega t + \delta), \textrm{~~where } \delta=\tan^{-1}\frac{B}{A}.
\end{equation*}
Rearranging the stress terms and considering times $t>0^+$, we get:
\begin{equation*}
    \sigma(t) = -E\varepsilon_0 - \varepsilon_0 \sqrt{\omega^2 \eta^2 + E^2} \sin\left( \omega t + \tan^{-1} \frac{\omega \eta}{E}\right).
\end{equation*}
Thus, $\tan \delta = \omega\eta/E$.

(c+d) Now, we'll consider the case where we prescribe the stress function $\sigma(t)$. 
The stress here will be $\sigma(t) = -\sigma_0 \mathcal{H}(t) - \sigma_0 \sin \omega t$, which leads to a differential form of: 
\begin{align*}
     \eta \dot{\varepsilon}(t) + E \varepsilon(t) = -\sigma_0  - \sigma_0 \sin \omega t. 
\end{align*}
Now, divide both sides by $\eta$ and solve this using integrating factors.
\begin{align*}
    \dot{\varepsilon} e^{E t/\eta} + \frac{E}{\eta}\varepsilon e^{E t/\eta} &= \frac{1}{\eta}e^{Et/\eta}\left[-\sigma_0 - \sigma_0 \sin \omega t\right]\\
    \frac{d}{dt}\left( \varepsilon e^{Et/\eta} \right) &= \frac{1}{\eta}e^{Et/\eta}\left[-\sigma_0 - \sigma_0 \sin \omega t\right]\\
    (\varepsilon(t) - \varepsilon(0^-))e^{Et/\eta} &= \int_{0^-}^t \frac{1}{\eta}e^{E\tilde{t}/\eta}\left[-\sigma_0 - \sigma_0 \sin \omega \tilde{t}\right] d\tilde{t}.
\end{align*}
I'll note here that the integral of an exponential multiplied by a sine function gives:
\begin{equation*}
    \int e^{at} \sin \omega t dt = \frac{e^{at}}{a^2 + \omega^2}(a \sin \omega t - \omega \cos \omega t) + \mathcal{C},
\end{equation*}
giving us:
\begin{align*}
 (\varepsilon(t) - \varepsilon(0^-))e^{Et/\eta} &= -\frac{\sigma_0}{\eta} \frac{e^{Et/\eta}}{\left( \frac{E}{\eta}\right)^2 + \omega^2}\left(\frac{E}{\eta} \sin \omega t - \omega \cos \omega t\right)\Bigg|_{0^-}^t - \frac{\sigma_0}{E} e^{Et/\eta}\\
 \varepsilon(t) &= -\frac{\sigma_0}{\eta} \frac{1}{\left( \frac{E}{\eta}\right)^2 + \omega^2}\left(\frac{E}{\eta} \sin \omega t - \omega \cos \omega t - \omega e^{-Et/\eta}\right)  - \frac{\sigma_0}{E}. 
\end{align*}
Now we take the exact same approach as above, which leads us to $\tan \delta = -\omega\eta/E$. 
This essentially says that stress \textcolor{red}{leads} the strain response by $\delta$ either way; we have kind of defined $\tan \delta$ here at strain with respect to stress, which should be the reverse of stress with respect to strain. 
Thus, $\tan \delta$ is the same for both.

