\setcounter{section}{1} 
\section*{Solutions to Examples II: Kinetics, Constitutive Laws, and Viscoelasticity I}
\label{soln-PS2}

\medskip
\subsection*{2--1. \textbf{Balance of mass} [4 pts].} 
We can start with the general form of the conservation of mass equation:
\begin{equation}
\label{eq:continuity}
    \frac{\partial \rho}{\partial t} + \bm{\nabla}_{\bm{x}} \rho \cdot\bm{v} + \rho \bm{\nabla}_{\bm{x}} \cdot \bm{v} = 0.
\end{equation}
    Now we want to incorporate the gradient in spherical coordinates, which will give us a few terms for each of the second and third terms here. 
    The gradient of a scalar function in spherical coordinates is:
\begin{equation*}
    \bm{\nabla}_{\bm{x}} \rho = \frac{\partial \rho}{\partial r} \bm{e}_r + \frac{1}{r} \frac{\partial \rho}{\partial \theta} \bm{e}_\theta  + \frac{1}{r\sin\theta}\frac{\partial \rho}{\partial \phi} \bm{e}_\phi  
\end{equation*}
The divergence of a vector function $\bm{v}$ is:
\begin{equation*}
    \bm{\nabla}_{\bm{x}} \cdot \bm{v} = \frac{\partial v_r}{\partial r} + 2\frac{v_r}{r}+ \frac{1}{r}\frac{\partial v_\theta}{\partial \theta} + \frac{1}{r \sin\theta}\frac{\partial v_\phi}{\partial \phi} + \frac{v_\theta}{r}\cot\theta.
\end{equation*}
Now, since $\bm{v} = v_r \bm{e}_r$, we can drop all of the partial derivatives with respect to $\theta$ and $\phi$, which leaves the divergence of velocity as
\begin{equation*}
    \bm{\nabla}_{\bm{x}} \cdot \bm{v} = \frac{\partial v_r}{\partial r} + 2\frac{v_r}{r}.
\end{equation*}
Incorporating incompressibility means that the density $\rho$ cannot be a function of $\bm{x}$ or $t$, so the time derivatives and gradient of $\rho$ corresponding to the first two terms, respectively, in Eq. \ref{eq:continuity} must be zero. 
This leaves
\begin{equation*}
    \frac{\partial v_r}{\partial r} + 2\frac{v_r}{r} = 0,
\end{equation*}
which we can separate and integrate from the bubble wall to a position with a corresponding velocity as follows:
\begin{align*}
    \frac{\partial v_r}{\partial r} &= -2\frac{v_r}{r}\\
    \int_{\dot{R}}^{v_r} \frac{1}{\tilde{v}_r} d\tilde{v}_r &= -2\int_R^r \frac{1}{\tilde{r}} d\tilde{r}\\
    \ln\left(\frac{v_r}{\dot{R}}\right) &= -2 \ln\left(\frac{r}{R}\right)\\
    v_r &= \frac{R^2}{r^2}\dot{R} ~.//
\end{align*}


\medskip
\subsection*{2--2. \textbf{Balance of momenta} [4 pts].}

To start, let's write out the balance of linear momentum.
\begin{equation*}
    \int_{\partial\Omega} \bm{t}~dA + \int_{\Omega} \bm{b} ~dV = \int_{\Omega} \rho \bm{a} ~dV.
\end{equation*}
From here, we can calculate each of the individual integrals on the left hand side. 
The first will benefit from the divergence theorem, which can turn it from a surface integral to a simple volumetric one of the divergence of the stress, which is:
\begin{equation*}
    \bm{t} = \bm{n} \cdot \bm{\sigma} \Rightarrow \bm{\sigma} = -\rho_w g x_3 \bn{I}.
\end{equation*}
The stress can then be used in the volume integral:
\begin{equation*}
    \int_{\partial\Omega} \bm{t}~dA  = \int_{\Omega} \bm{\nabla}_{\bm{x}} \cdot \bm{\sigma} ~dV.
\end{equation*}
As the stress is just a function of $x_3$, the dot product with the gradient operator simplifies to a derivative with respect to $x_3$ only, and the integrand evaluates to:
\begin{equation*}
\bm{\nabla}_{\bm{x}} \cdot \bm{\sigma} = \frac{\partial }{\partial x_3} \left( -\rho_w g x_3 \right) = -\rho_w g.
\end{equation*}
Integrating this constant stress divergence over the volume gives us a total integrated traction of $-4\pi\rho_w g R^3/3$. 
The body force integral is a bit more straightforward as it's just the gravitational force, but we should include the gradient in $x_2$, which is $\rho(\bm{x}) = \rho_w \left(1 + \frac{1}{2}\frac{x_2}{R}\right)$. 
We can be clever with our choice of the primary axis of our sphere to coincide with the direction of $\bm{e}_2$, which makes $x_2 = r\cos\phi$ and $x_1$ and $x_3$ some combination of $\pm r\cos\theta\sin\phi$ and $\pm r\sin\theta\sin\phi$. 
If we want both to be positive, we take $x_1 = r\cos\theta\sin\phi$ and $x_3 = r\sin\theta\sin\phi$. 
Integrating the body force in spherical coordinates (more important for later than now) then becomes
\begin{align*}
    \int_{\Omega} \bm{b} ~dV &= \int_0^\pi \int_0^R \int_0^{2\pi} \rho_w \left(1+\frac{1}{2}\cos\phi\right) g~r^2 \sin\phi ~d\theta dr d\phi\\
    &= 2\pi\rho_w g \int_0^\pi \int_0^R  \left(1+\frac{1}{2}\cos\phi\right)~r^2 \sin\phi ~ dr d\phi\\
    &= 2\pi \frac{R^3}{3}\rho_w g \left.\left(-\cos\theta -\frac{\cos^2\theta}{4}  \right)\right|_0^{\pi}\\
    &= \frac{4\pi R^3}{3}\rho_w g
\end{align*}
The body force is the weight due to gravity, which perfectly offsets the traction on the region occupied by the sphere supplied by the displaced water. 
Thus, this is naturally in equilibrium!
Note though, that this is ``for now''. 
If the sphere were to rotate we'd have to change the direction of the gradient, replacing the $\cos\theta/2$ with a different Cartesian direction's representation in spherical. 
More on this in a bit!

Now we want to consider the integral of 

