%% 
%% Copyright 2007-2025 Elsevier Ltd
%% 
%% This file is part of the 'Elsarticle Bundle'.
%% ---------------------------------------------
%% 
%% It may be distributed under the conditions of the LaTeX Project Public
%% License, either version 1.3 of this license or (at your option) any
%% later version.  The latest version of this license is in
%%    http://www.latex-project.org/lppl.txt
%% and version 1.3 or later is part of all distributions of LaTeX
%% version 1999/12/01 or later.
%% 
%% The list of all files belonging to the 'Elsarticle Bundle' is
%% given in the file `manifest.txt'.
%% 
%% Template article for Elsevier's document class `elsarticle'
%% with harvard style bibliographic references

\documentclass[preprint,12pt,authoryear]{elsarticle}

\usepackage[utf8]{inputenc}
\usepackage[margin=2cm]{geometry}
\usepackage{graphicx}
\usepackage{multirow}
\usepackage{amssymb}
\usepackage{amsmath}
\usepackage{setspace}
\usepackage{outlines}
\usepackage{enumitem}
\usepackage{xcolor}
\usepackage{upgreek}
\usepackage{mathabx}
\usepackage{algorithm}
\usepackage{algorithmic}
\usepackage{amsthm}
\usepackage[labelfont=bf,font=small]{caption}
\usepackage{epsfig}
\usepackage{geometry}
\usepackage{subfigure}
\usepackage[textsize=tiny]{todonotes}
\usepackage[normalem]{ulem}
\usepackage{lipsum}
\usepackage{array}
\usepackage{booktabs}
\usepackage{lineno}
\usepackage{array}
\usepackage{tikz}
\usepackage[english]{babel}

\usepackage[]{siunitx}
\sisetup{range-units=single,separate-uncertainty = true,print-unity-mantissa=false,per-mode=symbol,range-phrase = \text{--},
inter-unit-product=\cdot
}

\usepackage[
    protrusion=true,
    activate={true,nocompatibility},
    final,
    tracking=true,
    kerning=true,
    spacing=true,
    factor=1100]{microtype}
    
\SetTracking{encoding={*}, shape=sc}{40}

\usepackage{outlines}
\usepackage{enumitem}

\definecolor{lightblue}{rgb}{0.63, 0.74, 0.78}
\definecolor{seagreen}{rgb}{0.18, 0.42, 0.41}
\definecolor{orange}{rgb}{0.85, 0.55, 0.13}
\definecolor{silver}{rgb}{0.69, 0.67, 0.66}
\definecolor{rust}{rgb}{0.72, 0.26, 0.06}
\definecolor{purp}{RGB}{68, 14, 156}

\definecolor{zblue}{RGB}{8,81,156}
\definecolor{zpurp}{RGB}{84,39,143}
\definecolor{zred}{RGB}{165,15,21}

\colorlet{lightrust}{rust!50!white}
\colorlet{lightorange}{orange!25!white}
\colorlet{lightlightblue}{lightblue}
\colorlet{lightsilver}{silver!30!white}
\colorlet{darkorange}{orange!75!black}
\colorlet{darksilver}{silver!65!black}
\colorlet{darklightblue}{lightblue!65!black}
\colorlet{darkrust}{rust!85!black}
\colorlet{darkseagreen}{seagreen!85!black}



\usepackage{hyperref}
\hypersetup{
  colorlinks=true,
}
\usepackage{tabularx}
\usepackage{bbm}
\usepackage{bm}

\usepackage[nameinlink]{cleveref}
\crefname{equation}{}{}
\def\appendixname{}
\crefname{appendix}{}{}


\usepackage{setspace}
% \doublespacing
\setlength{\heavyrulewidth}{1.5pt}
% \setlength{\abovetopsep}{4pt}
 
\usepackage{soul}
\sethlcolor{yellow}

\usepackage[parfill]{parskip}

\usepackage{lineno}
\usepackage{tcolorbox}
%\linenumbers

\input{mathsymbols}

%% Use the option review to obtain double line spacing
%% \documentclass[authoryear,preprint,review,12pt]{elsarticle}

%% Use the options 1p,twocolumn; 3p; 3p,twocolumn; 5p; or 5p,twocolumn
%% for a journal layout:
%% \documentclass[final,1p,times,authoryear]{elsarticle}
%% \documentclass[final,1p,times,twocolumn,authoryear]{elsarticle}
%% \documentclass[final,3p,times,authoryear]{elsarticle}
%% \documentclass[final,3p,times,twocolumn,authoryear]{elsarticle}
%% \documentclass[final,5p,times,authoryear]{elsarticle}
%% \documentclass[final,5p,times,twocolumn,authoryear]{elsarticle}

%% For including figures, graphicx.sty has been loaded in
%% elsarticle.cls. If you prefer to use the old commands
%% please give \usepackage{epsfig}

%% The amssymb package provides various useful mathematical symbols
%\usepackage{amssymb}
%% The amsmath package provides various useful equation environments.
%\usepackage{amsmath}
%% The amsthm package provides extended theorem environments
%% \usepackage{amsthm}

%% The lineno packages adds line numbers. Start line numbering with
%% \begin{linenumbers}, end it with \end{linenumbers}. Or switch it on
%% for the whole article with \linenumbers.
%% \usepackage{lineno}

\journal{ME517---Mechanics of Soft Materials}

\setlength{\marginparwidth}{2cm}
\begin{document}

\begin{frontmatter}

\title{MECHENG 517---Mechanics of Soft Materials} %% Article title

%% Replace my name with yours!
\author{Prof. Jon Estrada} 


\affiliation{organization={University of Michigan},%Department and Organization
            addressline={2350 Hayward St.}, 
            city={Ann Arbor},
            postcode={48105}, 
            state={MI},
            country={USA}}

%% Abstract
% \begin{abstract}
% %% Text of abstract
% Abstract text.
% \end{abstract}

%%Graphical abstract
% %\includegraphics{grabs}
% \begin{graphicalabstract}
% \end{graphicalabstract}

%%Research highlights
% \begin{highlights}
% \item Research highlight 1
% \item Research highlight 2
% \end{highlights}

%% Keywords
% \begin{keyword}
%% keywords here, in the form: keyword \sep keyword

%% PACS codes here, in the form: \PACS code \sep code

%% MSC codes here, in the form: \MSC code \sep code
%% or \MSC[2008] code \sep code (2000 is the default)

% \end{keyword}

\end{frontmatter}

%% If you want to include line numbers, uncomment the line below this:
% \linenumbers

\section*{Proposing a Research Topic in the Mechanics of Soft Materials}

As your major deliverable this semester, you are going to develop a ``white paper'' proposal for a research project in the area of soft material mechanics. 
This is deliberately a bit different from a standard report by focusing on being forward-looking. 
A white paper is a $3-5$ page document that contains:
\begin{enumerate}
    \item A short \textbf{overview} of your topic
    \item A brief summary of the \textbf{current}, or state-of-the-art of \textbf{knowledge} in that area
    \item The \textbf{knowledge gap} you are interested in filling
    \item The \textbf{long-term goal} of what your project would do in the field if successful
    \item The testable \textbf{central hypothesis} governing the work
    \item Three \textbf{specific aims} you would pursue with their own working hypotheses
    \item \textbf{Preliminary data} backing up why you formulated and believe those hypotheses
    \item The \textbf{expected outcomes} or products of your proposed research that link back to the specific aims  
\end{enumerate}

This project development will occur over the course of the entire semester, and every assignment you have this semester will have a component that will develop this project proposal in some way. 
The schedule will be as follows:
\smallskip

\footnotesize
\begin{tabularx}{\textwidth}{ccXX}

\textbf{Checkpoint} & \textbf{Due Date} & \textbf{Major Component(s)} & \textbf{White Paper Section(s)} \\
\hline
\hline
P1 & Sept 17 & Topic ID and overview & Overview, long-term goal \\
\hline
P2 & Oct 1 & Literature review & Current knowledge and gap \\
\hline
P3 & Oct 15 & Data gathering and summary & Preliminary data \\
\hline
P4 & Nov 5 & Data analysis and & Central hypothesis \\
 &  & central hypothesis & Support from preliminary data \\
\hline
P5 & Nov 19 & Three specific aims and & Specific aims \\
 & & supported working hypotheses & Expected outcomes \\
\hline
P6 & Dec 8 & Full white paper draft (Peer Review) & All sections \\
 &  & (and peer review) &  \\
\hline
-- & Dec 15--16 & Oral presentations & Final draft \\
\hline
\end{tabularx}
\normalsize

\newpage


% Additional notes sections for ME517

\section*{Continuum Mechanics Notation} 

The following is a brief primer on my notation in solid mechanics. 
I'll generally follow the notation of \citet{holzapfelNonlinearSolidMechanics2002}. 
 
In my printed notes we will use lowercase \textit{italic} letters for scalars, \textit{\textbf{bold italic}} letters to denote first-rank tensor (i.e., vector) quantities, \textbf{bold upright} letters to denote second-rank tensor quantities, and blackboard Latin letters (like those used in set notation) for fourth-order tensors. 
The quantities in this form are denoted irrespective of basis, and are in ``direct'', or symbolic notation. 

\noindent I will use the following fonts to distinguish ranks of tensor in direct notation out of notational convenience:
\begin{eqnarray*}
a, b, \alpha, \beta, ... (\textrm{scalars}) & ~~~ \bm{a}, \bm{b}, \bm{A}, \bm{B}, ... (\textrm{vectors})\\
\mathbf{A}, \mathbf{B}, \mathbf{C}, ...(\textrm{2nd-rank tensors}) & ~~~ \mathbbm{A}, \mathbbm{C}, \mathbbm{I},... (\textrm{4th-rank tensors})
\end{eqnarray*}
    
When we aim to understand relationships in continuum mechanics often it is helpful to approach expressions in a component-by-component manner. 
In general I'll use index notation for the components of a tensor quantity in a typically-Cartesian basis $\{\bm{e}^{(i)}\}$. 
In this basis, the components of a second-rank tensor $\bn{A}$ are:
\begin{equation*}
    [\bn{A}]^{\bm{e}} = \begin{bmatrix}
A_{11}^{\bm{e}}  & A_{12}^{\bm{e}}  & A_{13}^{\bm{e}} \\
A_{21}^{\bm{e}}  & A_{22}^{\bm{e}}  & A_{23}^{\bm{e}} \\
A_{31}^{\bm{e}}  & A_{32}^{\bm{e}}  & A_{33}^{\bm{e}} 
\end{bmatrix}.
\end{equation*}

Alternately, we can express $\bn{A}$ as
\begin{equation*}
    \bn{A} = A_{ij}^{\bm{e}} \bm{e}^{(i)} \otimes \bm{e}^{(j)}
\end{equation*}
If we're using just one basis, it's common to drop the additional superscript on $A_{ij}^{\bm{e}}$, leaving $A_{ij}$, but it's important to remember that the components $A_{ij}$ depend on the basis we pick.
It's also rather common to use the notation $\bm{e}_i$ rather than $\bm{e}^{(i)}$, which is equivalent as long as we do not need to change bases using the change-of-basis tensor $\bn{Q}$. 
If you do want to use $\bm{e}_i$ and $\bm{f}_j$ when changing bases, it's important to remember that $\bm{f}_j$ represents the first of three vectors and not the $j$th component of a single vector $\bm{f}$. 





% \section{Full White Paper Draft}

%This is just a placeholder for now
% \section{Specific Aims and Working Hypotheses}

%This is just a placeholder for now

% \section{Data Analysis and Central Hypothesis}

%This is just a placeholder for now

% \section{Data Gathering and Summary}

%This is just a placeholder for now

% \include{part2}
\section*{Project I: Topic ID and Overview (due Sept 17)}

%This is just a placeholder for now

The first step in your semester-long research proposal development is to select a topic area in the mechanics of soft materials that's sufficiently interesting to you. 
It may be helpful to think of the proposal-writing process as the following. 
You want to study something that you are especially interested in, but you don't yet have the resources that you need to pursue this fully. 
Your job is (eventually) to communicate what it is you want to study, why it's worthwhile to be studied, and enumerate all of the reasons it's in some benefactor's interest to provide you the support that you need.
The particular benefactor we will leverage is the National Science Foundation, which cares about making fundamental ``vertical'' advances in fields (known as their \textit{Intellectual Merit} criterion) and having their funded projects improve society (known as the \textit{Broader Impact} to society criterion). 

Aim for approximately 500 words of total text, such as to reflect the important three Cs: \textbf{\textit{clear}}, \textbf{\textit{concise}}, and \textbf{\textit{compelling}}.
Your submission should be structured in three sections as separated below, and should address the following points: 

\begin{enumerate}
\item \textbf{Statement of Research Interest (why you personally want to study the subject)}
\begin{itemize}
\item Describe an area or phenomenon in the mechanics of soft materials that you find compelling. 
\item What motivates your interest and pursuit of this subject (e.g., your current or developing expertise, research interests, or otherwise)? \textit{Note: This may be more personal or anecdotal and is for my own understanding of your topic selection!}
\end{itemize}
\item \textbf{Intellectual Merit (why it is objectively worth delving deeper)}
\begin{itemize}
\item Describe, to someone with expertise in mechanics but perhaps not your system of interest, the core scientific principles underpinning (or perhaps, enabling development in) your topic of interest. 
\item Given the course syllabus, how will particular material we will cover this semester relate to what you propose? What background information do you need to do not just a good, but great, job in proposing something interesting? 
\end{itemize}
\item \textbf{Broader Impact (who, or what, does studying this area benefit?)}
\begin{itemize}
\item How might advances you envision in this area be impactful beyond your own interest? 
\item What does a ``winning scenario'' in this area look like? Briefly describe who might benefit (e.g., particular industries, health/science sectors, the public) and how that could plausibly happen.
\end{itemize}
\end{enumerate}

\emph{A strong submission will clearly illustrate your personal goals with this project, and show how your interest could manifest as advances in the broader field and society.}








%% The appendix will contain the example problems assigned as problem sets. 
\appendix

%include{PS5}
%\include{PS4}
%\include{PS3}
%
\section{Continuum Mechanics Fundamentals}
\label{PS2}

This set of example problems is due on October 1, 2025. 

% This is a placeholder for the example problems from the second problem set. 
% You'll replace this file with the one I supply on canvas. 
\include{PS1}


%% For citations use: 
%%       \citet{<label>} ==> Lamport (1994)
%%       \citep{<label>} ==> (Lamport, 1994)
%%
%Example citation, See \citet{lamport94}.

%% If you have bib database file and want bibtex to generate the
%% bibitems, please use
%%
%%  \bibliographystyle{elsarticle-harv} 
%%  \bibliography{<your bibdatabase>}

%% else use the following coding to input the bibitems directly in the
%% TeX file.

%% Refer following link for more details about bibliography and citations.
%% https://en.wikibooks.org/wiki/LaTeX/Bibliography_Management

%\begin{thebibliography}{00}

%% For authoryear reference style
%% \bibitem[Author(year)]{label}
%% Text of bibliographic item

\bibliographystyle{elsarticle-harv} 
\bibliography{cas-refs}

%\end{thebibliography}
\end{document}

\endinput
%%
%% End of file `elsarticle-template-harv.tex'.


