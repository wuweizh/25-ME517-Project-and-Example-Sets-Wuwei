
\section*{Examples I. Mathematical Preliminaries (due \textcolor{red}{Sept 19})}
\textcolor{red}{(Rev note: v2)}
\label{PS1}

This set of example problems is due on September 17, 2025. 
I request that you type up your responses in \LaTeX~ rather than write them out by hand. 
The primary reason is to become better acquainted with writing up mechanics in archival format. 
If you have diagrams, plots, etc., please add them as attached figures using the \texttt{includegraphics} command. 

\bigskip
\subsection*{1--1. \textbf{Convolutional integrals} [4 pts].} The response of a 1D viscoelastic material to an applied forcing function is given using a convolutional integral:
\begin{equation}
    \varepsilon(t) = \int_0^t J(t-\tau) \frac{d\sigma(\tau)}{d\tau} d\tau,
\end{equation}
where $\varepsilon(t)$ is the time-dependent strain response, $\sigma(t)$ is the prescribed stress function, and $J(t)$ is the material compliance, assumed to not depend on the level of stress applied. 
Say we have a compliance function 
\begin{equation}
    J(t) = J_\infty + (J_0-J_\infty)\exp[-t/\tau_c],
\end{equation}
where $J_0, J_\infty, \tau_c$ are all constants $\in \mathbbm{R}$. 
We subject this material to two different loading profiles: (a) step load $\sigma_1(t) = \sigma_0 H(t)$ and (b) a sinusoidal load $\sigma_2(t) = \sigma_0  \sin(\omega t)$, where $\sigma_0$ and $\omega$ are also constant, and $H(t)$ is the step function. 

Determine the corresponding Laplace transforms of the strain functions $\mathcal{L}\{\varepsilon_1(t)\}$ and $\mathcal{L}\{\varepsilon_2(t)\}$. 

\textit{\textbf{Dare mode:}} If you have taken complex analysis, you can compute the inverse transform of polynomial forms because this type of model yields terms with simple poles (i.e. linear in $s$). 
The formulae for this (see e.g. \cite{rileyMathematicalMethodsPhysics2006} Chs. 24 and 25) are:
\begin{equation*}
    \textrm{Residue for simple poles: } R(f(s),s_0) = \lim\limits_{s\rightarrow s_0} \left[ (s-s_0) f(s) \right],
\end{equation*}
and you multiply each residue by the shift from zero, i.e.,
\begin{equation*}
    f(t) = \mathcal{L}^{-1}\{F(s)\} = \sum \left( \textrm{residues of } F(s)e^{s_0 t} \textrm{ at all poles } s_0 \right)
\end{equation*}
\textit{If you dare}, determine the corresponding strain histories $\varepsilon_1(t)$ and $\varepsilon_2(t)$. The first is relatively straightforward; the latter has complex poles and more terms. 

%\newpage
\bigskip
\textbf{Answer:}
Given
\begin{equation}
  \varepsilon(t)=\int_0^t J(t-\tau)\,\frac{d\sigma(\tau)}{d\tau}\,d\tau,
  \qquad
  J(t)=J_\infty+(J_0-J_\infty)e^{-t/\tau_c},
\end{equation}
its Laplace transform (with $\sigma(0^-)=0$ and using $\mathcal{L}\{f*g\}=F(s)G(s)$,
$\mathcal{L}\{\dot{\sigma}\}=s\varepsilon(s)$) is
\begin{equation}
  E(s)=\mathcal{L}\{\varepsilon(t)\}=J(s)\,s\,\varepsilon(s),
  \qquad
  J(s)=\frac{J_\infty}{s}+\frac{J_0-J_\infty}{\,s+1/\tau_c\,}.
\end{equation}

\paragraph{(a) Step load}
For $\sigma_1(t)=\sigma_0 H(t)$, we have $\varepsilon_1(s)=\sigma_0/s$, hence
\begin{equation}
  \,E_1(s)=\sigma_0\,J(s)=\sigma_0\!\left(\frac{J_\infty}{s}+\frac{J_0-J_\infty}{\,s+1/\tau_c\,}\right).\,
\end{equation}

\paragraph{(b) Sinusoid}
For $\sigma_2(t)=\sigma_0\sin(\omega t)$, we have $\varepsilon_2(s)=\dfrac{\sigma_0\omega}{s^2+\omega^2}$, hence
\begin{equation}
  \,E_2(s)=J(s)\,s\,\Sigma_2(s)
  =\sigma_0\omega\left[\frac{J_\infty}{s^2+\omega^2}
  +\frac{(J_0-J_\infty)\,s}{(s+1/\tau_c)(s^2+\omega^2)}\right].\,
\end{equation}

\bigskip
\subsection*{1--2. \textbf{Index notation} [4 pts].} Let $\bm{p}, \bm{q}, \bm{r}, \bm{a}, \bm{b}$ be vector fields on $\mathbbm{R}^3$ and $\bn{Q}$ be a change-of-basis tensor on $\mathbbm{R}^3$. Show the following identities to be true using index notation. 

\begin{itemize}
    \item $\bm{p} \times (\bm{q} \times \bm{r}) = (\bm{r} \cdot \bm{p}) \bm{q} - (\bm{q} \cdot \bm{p}) \bm{r}$
    \item $(\bm{p} \times \bm{q}) \cdot (\bm{a} \times \bm{b}) = (\bm{p} \cdot \bm{a}) (\bm{q} \cdot \bm{b}) - (\bm{q} \cdot \bm{a})(\bm{p} \cdot \bm{b})$
    \item $(\bm{a} \otimes \bm{b})(\bm{p} \otimes \bm{q}) = \bm{a}\otimes\bm{q}(\bm{b} \cdot \bm{p}) $
    \item $\bn{Q}^\intercal\bm{a} \cdot \bn{Q}^\intercal\bm{b} = \bm{a}\cdot\bm{b} $
\end{itemize}

\bigskip 
\textbf{Answer:}
Use $(\bm{a}\times\bm{b})_i=\varepsilon_{ijk}a_j b_k$ and the identity 
$\varepsilon_{ijk}\varepsilon_{imn}=\delta_{jm}\delta_{kn}-\delta_{jn}\delta_{km}$.

\paragraph{(i)}
\begin{align*}
\big[\bm{p}\times(\bm{q}\times\bm{r})\big]_i 
&= \varepsilon_{ijk}p_j(\varepsilon_{klm}q_l r_m) \\
&= (\delta_{i\ell}\delta_{jm}-\delta_{im}\delta_{j\ell})\,p_j q_\ell r_m \\
&= (p_j r_j)q_i-(p_j q_j)r_i \\
&=((\bm{r}\cdot\bm{p})\,q_i - (\bm{q}\cdot\bm{p})\,r_i).
\end{align*}

\paragraph{(ii)}
\begin{align*}
(\bm{p}\times\bm{q})\cdot(\bm{a}\times\bm{b})
&=\varepsilon_{ijk}p_j q_k\,\varepsilon_{ilm}a_l b_m \\
&=(\delta_{jl}\delta_{km}-\delta_{jm}\delta_{kl})\,p_j q_k a_l b_m \\
&=p_j a_j q_k b_k-p_j b_j q_k a_k\\
&=(\bm{p}\cdot\bm{a})(\bm{q}\cdot\bm{b})-(\bm{p}\cdot\bm{b})(\bm{q}\cdot\bm{a}).
\end{align*}

\paragraph{(iii)}
\begin{align*}
\big[(\bm{a}\otimes\bm{b})(\bm{p}\otimes\bm{q})\big]_{ij}
&=(a_i b_k)(p_k q_j) \\
&=a_i q_j\,(b_k p_k) \\
&=\big(\bm{b}\cdot\bm{p}\big)\,(\bm{a}\otimes\bm{q})_{ij}.
\end{align*}

\paragraph{(iv)}
\begin{align*}
(\bn{Q}^\top\bm{a})\cdot(\bn{Q}^\top\bm{b})
&=(Q_{ki}a_i)(Q_{kj}b_j) \\
&=(Q_{ki}Q_{kj})\,a_i b_j \\
&=\delta_{ij}a_i b_j \\
&=\bm{a}\cdot\bm{b},
\end{align*}
where $\bn{Q}$ is an orthogonal change-of-basis tensor ($Q_{ki}Q_{kj}=\delta_{ij}$).
\bigskip

\subsection*{1--3. \textbf{Tensors and vectors} [4 pts].}
The second-order projection tensors $\bn{P}_{\bm{n}}^{||}$ and $\bn{P}_{\bm{n}}^{\perp}$ are useful operators that take a vector $\bm{u}$ and map that vector to its part parallel and perpendicular to a vector $\bm{n}$, respectively. 

They are defined via:
\begin{equation*}
    \bm{u}_{||} = (\bm{u} \cdot \bm{n}) \bm{n} = (\bm{n} \otimes \bm{n}) \bm{u} = \bn{P}_{\bm{n}}^{||} \bm{u},
\end{equation*}
\begin{equation*}
    \bm{u}_{\perp} = \bm{u} - \bm{u}_{||} = (\bn{I} - \bm{n} \otimes \bm{n}) \bm{u} = \bn{P}_{\bm{n}}^{\perp} \bm{u}.
\end{equation*}

The projection tensors have properties
\begin{align*}
    \bn{P}_{\bm{n}}^{||} + \bn{P}_{\bm{n}}^{\perp} &= \bn{I} \\
    \left(\bn{P}_{\bm{n}}^{||} \right)^m &= \bn{P}_{\bm{n}}^{||} ~\forall ~m \in \mathbbm{Z}^+\\
    \left(\bn{P}_{\bm{n}}^{\perp} \right)^m &= \bn{P}_{\bm{n}}^{\perp} ~\forall ~m \in \mathbbm{Z}^+\\
    \bn{P}_{\bm{n}}^{||} \bn{P}_{\bm{n}}^{\perp} = \bn{P}_{\bm{n}}^{\perp} \bn{P}_{\bm{n}}^{||}  &= \bn{0}
\end{align*}

Using the projection tensors, show that $\bm{u} = (\bm{u} \cdot \bm{n}) \bm{n} + \bm{n} \times (\bm{u} \times \bm{n} )$.
\bigskip
\textbf{Answers:}
Using the vector triple–product identity
$\bm{n}\times(\bm{u}\times\bm{n})=\bm{u}(\bm{n}\cdot\bm{n})-\bm{n}(\bm{u}\cdot\bm{n})$,
and assuming $\|\bm{n}\|=1$, we obtain
$\bm{n}\times(\bm{u}\times\bm{n})=\bm{u}-(\bm{u}\cdot\bm{n})\bm{n}$.
Hence,
$\bm{u}=(\bm{u}\cdot\bm{n})\,\bm{n}+\bm{n}\times(\bm{u}\times\bm{n})$
as required.

\bigskip
\subsection*{1--4. \textbf{Vector and tensor calculus} [4 pts].} Show the following vector and tensor identities to be true using index notation:

\begin{itemize}
    \item $\gradX \times (\phi \bm{a}) = \phi \gradX \times \bm{a} + (\gradX\phi) \times \bm{a}$
    \item $\gradX (\bm{a} \cdot \bm{b}) = (\bm{a} \cdot \gradX) \bm{b} + (\bm{b} \cdot \gradX) \bm{a} + \bm{a} \times (\gradX \times \bm{b}) + \bm{b} \times (\gradX \times \bm{a})$
    \item $ (\bn{A} \bn{B}) \bn{:} \bn{C} = (\bn{A}^\intercal \bn{C})\bn{:} \bn{B} = (\bn{C} \bn{B}^\intercal)\bn{:} \bn{A}$
    \item Let $J = \det \bn{F}$. Show\footnote{It will help to use the expression for the determinant of a tensor in index notation!} that $\frac{\partial J}{\partial \bn{F}} = J \bn{F}^{-\intercal}$. 
    \end{itemize}

%\newpage
\bigskip
\textbf{Answer:}
\paragraph{(i)}
\begin{align*}
\big[\nabla \times (\phi \bm{a})\big]_i
&= \varepsilon_{ijk}\,\partial_j(\phi a_k) \\
&= \varepsilon_{ijk}\,\big((\partial_j \phi)a_k + \phi\,\partial_j a_k\big) \\
&= \big[(\nabla \phi)\times \bm{a}\big]_i + \big[\phi\,(\nabla \times \bm{a})\big]_i,
\end{align*}
so
$\nabla \times (\phi \bm{a}) = \phi (\nabla \times \bm{a}) + (\nabla \phi)\times \bm{a}$.

\paragraph{(ii)}
\begin{align*}
\partial_i(\bm{a}\cdot\bm{b})
&=\partial_i(a_m b_m)=(\partial_i a_m)b_m+a_m(\partial_i b_m).
\end{align*}
Use the triple–product expansion in index form
$\big[\bm{a}\times(\nabla\times\bm{b})\big]_i
=\varepsilon_{ijk}a_j\,\varepsilon_{k\ell m}\,\partial_\ell b_m
=(\delta_{i\ell}\delta_{jm}-\delta_{im}\delta_{j\ell})\,a_j\partial_\ell b_m
= a_m\,\partial_i b_m-(\bm{a}\cdot\nabla)b_i$,
and similarly
$\big[\bm{b}\times(\nabla\times\bm{a})\big]_i
= b_m\,\partial_i a_m-(\bm{b}\cdot\nabla)a_i$.

Adding these two gives
$\big[\bm{a}\times(\nabla\times\bm{b})\big]_i
+\big[\bm{b}\times(\nabla\times\bm{a})\big]_i
= \partial_i(a_m b_m)-(\bm{a}\cdot\nabla)b_i-(\bm{b}\cdot\nabla)a_i$.
Rearranging,
$\nabla(\bm{a}\cdot\bm{b})
=(\bm{a}\cdot\nabla)\bm{b}+(\bm{b}\cdot\nabla)\bm{a}
+\bm{a}\times(\nabla\times\bm{b})+\bm{b}\times(\nabla\times\bm{a})$

\paragraph{(iii)}
\begin{align*}
(\bm{A}\bm{B}):\bm{C} &= A_{ik}B_{kj}C_{ij}, \\
(\bm{A}^\top \bm{C}):\bm{B} &= (A^\top)_{ki}C_{ij}B_{kj} = A_{ik}C_{ij}B_{kj}, \\
(\bm{C}\bm{B}^\top):\bm{A} &= C_{ij}(B^\top)_{jk}A_{ik} = C_{ij}B_{kj}A_{ik}.
\end{align*}
All are the same scalar, so
$(\bm{A}\bm{B}):\bm{C}=(\bm{A}^\top \bm{C}):\bm{B}=(\bm{C}\bm{B}^\top):\bm{A}$.

\paragraph{(iv)}
By definition,
$$
J \;=\; \frac{1}{6}\,\varepsilon_{ijk}\,\varepsilon_{pqr}\,F_{ip}\,F_{jq}\,F_{kr}.
$$

Differentiate $J$ with respect to $F_{ab}$. By the product rule, there are three identical terms (because the expression is symmetric in the three $F$'s), hence:
$$
\frac{\partial J}{\partial F_{ab}}
\;=\;
\frac{1}{6}\,\varepsilon_{ijk}\,\varepsilon_{pqr}\,
\Big(\delta_{ia}\delta_{pb}\,F_{jq}F_{kr}
\;+\;
F_{ip}\,\delta_{ja}\delta_{qb}\,F_{kr}
\;+\;
F_{ip}\,F_{jq}\,\delta_{ka}\delta_{rb}\Big).
$$

Relabel dummy indices in the second and third terms to match the first-term pattern and combine them to obtain:
$$
\frac{\partial J}{\partial F_{ab}}
\;=\;
\frac{1}{2}\,\varepsilon_{a j k}\,\varepsilon_{b q r}\,F_{jq}\,F_{kr}.
$$

Define the cofactor matrix $\operatorname{cof}\bm{F}$ with entries
$$
(\operatorname{cof}\bm{F})_{ba}
\;:=\;
\frac{1}{2}\,\varepsilon_{a j k}\,\varepsilon_{b q r}\,F_{jq}\,F_{kr}.
$$

Thus,
$$
\frac{\partial J}{\partial F_{ab}}
\;=\;
(\operatorname{cof}\bm{F})_{ba}.
$$

Next, show that this cofactor matrix satisfies the adjugate identity. Compute:
$$
F_{ia}\,(\operatorname{cof}\bm{F})_{ba}
\;=\;
F_{ia}\left(\frac{1}{2}\,\varepsilon_{a j k}\,\varepsilon_{b q r}\,F_{jq}\,F_{kr}\right)
\;=\;
\frac{1}{2}\,\varepsilon_{a j k}\,F_{ia}\,\varepsilon_{b q r}\,F_{jq}\,F_{kr}.
$$

Using the identity
$$
\varepsilon_{a j k}\,F_{ia}\,F_{jq}\,F_{kr}
\;=\;
\varepsilon_{i q r}\,\det\bm{F}
\;=\;
\varepsilon_{i q r}\,J,
$$
we get
$$
F_{ia}\,(\operatorname{cof}\bm{F})_{ba}
\;=\;
\frac{1}{2}\,\varepsilon_{b q r}\,\big(\varepsilon_{i q r}\,J\big)
\;=\;
J\,\frac{1}{2}\,\varepsilon_{b q r}\,\varepsilon_{i q r}
\;=\;
J\,\delta_{bi}.
$$

Therefore,
$$
\bm{F}\,(\operatorname{cof}\bm{F})^\top \;=\; J\,\bm{I}
\qquad\Longrightarrow\qquad
(\operatorname{cof}\bm{F})^\top \;=\; J\,\bm{F}^{-1}.
$$

Taking transposes,
$$
\operatorname{cof}\bm{F} \;=\; J\,\bm{F}^{-\top}.
$$

Since $(\dfrac{\partial J}{\partial F_{ab}}=(\operatorname{cof}\bm{F})_{ba})$, we conclude
$$
\frac{\partial J}{\partial \bm{F}} \;=\; J\,\bm{F}^{-\top}
$$

\bigskip
\subsection*{1--5. \textbf{Kinematics} [8 pts].} The Happy Gelatinous Cube (HGC, pictued) $\mathcal{G}$ exists on a domain of $\{-1\leq X_1 , X_3\leq1, 0\leq X_2 \leq 2\}$ at initial time $t=0$. 
At all times, the bottom surface of the HGC does not move. 
Its top surface moves sinusoidally in time at frequency $\omega$ by a maximum magnitude of $\alpha$. 
At maximum compression, points in the centers of the surfaces defined by outward normals $\bm{e}_1$ and $\bm{e}_3$ experience maximum displacements of magnitude $\beta$. 

\medskip
(a) Determine the deformation gradient tensor $[\bn{F}(\bm{X})]^{\bm{e}}$ for all $\bm{X}\in \mathcal{G}$. 
Describe any assumptions you make about the shape of the HGC as it deforms. 

\medskip
(b) Determine the stretch magnitude of a small fiber positioned at a height $X_2 = 1$ and oriented at an angle $\theta$ from the $\bm{e}_1$ axis \textcolor{red}{(in either the $\bm{e}_1- \bm{e}_2$ or $\bm{e}_1- \bm{e}_3$ plane)}. 

\medskip
(c) Determine the Lagrange-Green strain tensor $\bn{E}$ and the material logarithmic strain tensor $\bn{E}_H = \ln (\bn{U})$ for the geometric center $\bm{X}_c$ of the HGC\footnote{Note that the log of a tensor is defined by writing it spectrally and replacing each eigenvalue with the log of that eigenvalue. For a case of no shear/off-diagonal terms, you can just take the log of each element on the diagonal to get $\ln(\bn{U})$.}. 
What are the maximum and minimum values of the strain eigenvalues $E_i(t)$ and $E_i^H(t)$? 
Would you expect one set to be more symmetric about zero as $\alpha$ gets large, and why?

\medskip
(d) Determine both the material point acceleration $\bm{A}(\bm{X}_1)$ at \textcolor{red}{\sout{, and spatial acceleration $\bm{a}(\bm{x}_1)$ of material moving through,}} a point $\frac{1}{2} \bm{e}_1 + 2\bm{e}_2 + \frac{1}{2} \bm{e}_3$.  

\begin{figure}
\centering
\animategraphics[loop,autoplay,width=4in]{10}{instr-figures/The_Happy_Gelatinous_Cube-}{1}{10}
\end{figure}
\textbf{Answer:}
We assume the following time-harmonic field, quadratic through thickness $X_2$, since the size of the top and bottom surfaces remains:
$$
x_1(\bm{X},t)=X_1-\beta\,X_2\,(2-X_2)\,X_1\,\sin(\omega t),\qquad
x_2(\bm{X},t)=X_2-\alpha\,\frac{X_2}{2}\,\sin(\omega t),\qquad
x_3(\bm{X},t)=X_3-\beta\,X_2\,(2-X_2)\,X_3\,\sin(\omega t).
$$

\bigskip
\paragraph{(a)}
\begin{equation*}
    [\bm{F}(\bm{X})]^{\bm{e}}=\partial\bm{x}/\partial\bm{X}
\end{equation*}
\medskip
Compute spatial derivatives (let $s=\sin(\omega t)$):
$$
\frac{\partial x_1}{\partial X_1} = 1-\beta X_2 (1-X_2) s, \qquad
\frac{\partial x_1}{\partial X_2} = 2\beta X_1 (X_2-1) s, \qquad
\frac{\partial x_1}{\partial X_3} = 0,
$$
$$
\frac{\partial x_2}{\partial X_1} = 0, \qquad
\frac{\partial x_2}{\partial X_2} = 1+\tfrac{\alpha}{2}s, \qquad
\frac{\partial x_2}{\partial X_3} = 0,
$$
$$
\frac{\partial x_3}{\partial X_1} = 0, \qquad
\frac{\partial x_3}{\partial X_2} = 2\beta X_3 (X_2-1) s, \qquad
\frac{\partial x_3}{\partial X_3} = 1-\beta X_2 (1-X_2) s.
$$
Therefore
$$
\bm{F}(\bm{X},t)=
\begin{bmatrix}
1-\beta X_2 (1-X_2) s & 2\beta X_1 (X_2-1) s & 0 \\
0 & 1+\dfrac{\alpha}{2}s & 0 \\
0 & 2\beta X_3 (X_2-1) s & 1-\beta X_2 (1-X_2) s
\end{bmatrix}.
$$

\bigskip
\paragraph{(b)}
We evaluate on the mid-height centerline to eliminate shear (set $X_2=1$, $X_1=0$, $X_3=0$). Then
$$
\bm{F}_c(t)=\mathrm{diag}\!\Big(1-\beta s,\; 1+\tfrac{\alpha}{2}s,\; 1-\beta s\Big),
\qquad
\bm{C}_c(t)=\bm{F}_c^\top\bm{F}_c=\mathrm{diag}\!\Big((1-\beta s)^2,\; (1+\tfrac{\alpha}{2}s)^2,\; (1-\beta s)^2\Big).
$$

For a unit material fiber $\bm{a}_0$ the stretch is $\lambda=\sqrt{\bm{a}_0\cdot\bm{C}_c\,\bm{a}_0}$.

\medskip
In the $\bm{e}_1$--$\bm{e}_3$ plane, take
$$
\bm{a}_0=\cos\theta\,\bm{e}_1+\sin\theta\,\bm{e}_3.
$$
Then
$$
\lambda^2(\theta,t)=(1-\beta s)^2\cos^2\theta+(1-\beta s)^2\sin^2\theta=(1-\beta s)^2.
$$

\bigskip
\paragraph{(c)}
At $\bm{X}_c$, we have the diagonal $\bm{F}_c$ above, hence
$$
\bm{C}_c=\mathrm{diag}\!\Big((1-\beta s)^2,\; (1-\tfrac{\alpha}{2}s)^2,\; (1-\beta s)^2\Big).
$$
The Lagrange–Green strain is
$$
\bm{E}=\tfrac{1}{2}(\bm{C}-\bm{I})
=\tfrac{1}{2}\,\mathrm{diag}\!\Big((1-\beta s)^2-1,\; (1-\tfrac{\alpha}{2}s)^2-1,\; (1-\beta s)^2-1\Big).
$$
The right stretch tensor is
$$
\bm{U}=\sqrt{\bm{C}}=\mathrm{diag}\!\Big(1-\beta s,\; 1-\tfrac{\alpha}{2}s,\; 1-\beta s\Big)
\;\; \text{(assuming amplitudes keep entries positive)}.
$$
Therefore the material logarithmic strain is
$$
\bm{E}_H=\ln\bm{U}
=\mathrm{diag}\!\Big(\ln(1-\beta s),\; \ln(1-\tfrac{\alpha}{2}s),\; \ln(1-\beta s)\Big).
$$

The strain eigenvalues are the diagonal entries. Maxima/minima occur at $s=\pm 1$:
$$
E_1(t)=E_3(t)=\tfrac{1}{2}\big((1-\beta s)^2-1\big),\qquad
E_2(t)=\tfrac{1}{2}\big((1-\tfrac{\alpha}{2}s)^2-1\big),
$$
$$
E_1^H(t)=E_3^H(t)=\ln(1-\beta s),\qquad
E_2^H(t)=\ln\!\big(1-\tfrac{\alpha}{2}s\big).
$$

\emph{Symmetry comment:} the set $\{E_i^H\}$ is more symmetric about zero because $\ln\lambda=-\ln(\lambda^{-1})$, whereas $E_i=\tfrac{1}{2}(\lambda_i^2-1)$ grows asymmetrically with large compression/tension. As $\alpha$ increases, this asymmetry in $\{E_i\}$ becomes more pronounced than in $\{E_i^H\}$.

\bigskip
\paragraph{(d)}
Evaluate the material acceleration, the second time derivative of point $\bm{x}(\bm{X}_1,t)=(\tfrac{1}{2},\;2+\alpha s,\;\tfrac{1}{2})$:
$$
\bm{A}(\bm{X},t)=\partial_t^2 \bm{x}(\bm{X},t).
$$
Since $\partial_t^2[\sin(\omega t)]=-\omega^2\sin(\omega t)$, we obtain
$$
\bm{A}(\bm{X}_1,t)=(0,\;-\omega^2\alpha\sin(\omega t),\;0).
$$
% This is a placeholder for the example problems from the first problem set. 
% You'll replace this file with the one I supply on canvas. 