\setcounter{section}{1} % This causes the next section to be Appendix B


\section*{Examples II: Kinetics, Constitutive Laws, and Viscoelasticity I}
\label{PS2}
\textcolor{red}{(Rev note: v3)}


This set of example problems is due on October 3, 2025. 
As before, I request that you type up your responses in \LaTeX~ rather than write them out by hand. 

\medskip
\subsection*{2--1. \textbf{Balance of mass} [4 pts].} 
A large piece of polydimethylsiloxane (PDMS) of uniform density $\rho(\bm{x},t)$ has a central spherical bubble of time-evolving radius $R(t)$, initial radius $R_0$, and wall velocity of $\dot{R}$. 
The hole is subject to a uniform surface traction in the $\bm{e}^{(r)} \equiv \bm{e}_{\bm{r}}$ direction from an axisymmetric pressure, and maintains spherical symmetry over time. 

\medskip
The position of a point in the material can be written as $\bm{x} = r(R,t) \bm{e}_{\bm{r}}$ with reference position $\bm{X} = r_0(R_0)$, while the velocity of that point can be written as $\bm{v}(r,t) = v_r \bm{e}_{\bm{r}}$.

\medskip
Using the conservation of mass equation, show that the material must satisfy
%\begin{equation}
%\rho_{,t} + (\rho v_i)_{,i} = 0,
%\end{equation}
\begin{equation*}
\rho_{,t}+ \rho_{,r} v_r + \frac{\rho}{r} (\textcolor{red}{2}v_r + r v_{r,r}) = 0,
\end{equation*}
and hence, show that an assumption of incompressibility for PDMS results in 
\begin{equation*}
v_r(r,t) = \frac{R^2 \dot{R}}{r^2}.
\end{equation*}
\medskip
\textbf{Answer:}
The Eulerian continuity equation is
$$
\frac{\partial \rho}{\partial t} + \nabla\!\cdot(\rho \bm v) = 0.
$$

For spherical symmetry with $\bm v(r,t) = v_r(r,t)\,\bm e_r$ and $\rho=\rho(r,t)$, we have
$$
\nabla \cdot (\rho v_r \bm e_r) = \frac{1}{r^2} \frac{\partial}{\partial r} \!\big( r^2 \rho v_r \big).
$$
Thus,
$$
\rho_{,t} + \frac{1}{r^2}\frac{\partial}{\partial r}(r^2 \rho v_r) = 0,
$$
or equivalently
$$
\rho_{,t} + \rho_{,r} v_r + \frac{\rho}{r}(2 v_r + r v_{r,r}) = 0.
$$

For incompressible PDMS ($\rho=$ constant), the condition reduces to
$$
v_{r,r} + \frac{2}{r} v_r = 0 \quad \Longrightarrow \quad r^2 v_r(r,t) = C(t).
$$
Using $v_r(R,t) = \dot R(t)$ at the bubble wall, $C(t) = R^2 \dot R$. Hence
$$
v_r(r,t) = \frac{R^2(t)\dot R(t)}{r^2}.
$$

\medskip
\subsection*{2--2. \textbf{Balance of momenta} [4 pts].} A spherical hydrogel body $\mathcal{B}$ with a linear density gradient is currently submerged in water as depicted in the figure. 
The sphere has coordinates $\bm{x}$ in a region $\Omega$ with position-dependent density $\rho(\bm{x})$. 

\begin{figure}[H]
\vspace{-2em}
\centering
\includegraphics[width=3in]{instr-figures/PS2-Q1.pdf}      
\caption{\small{Hydrogel sphere with a linear density gradient submerged in water. The water has a density $\rho_w$, while the sphere has a density at its leftmost point of $\rho_w/2$ and at its rightmost point of $3\rho_w/2$.}}
\end{figure}

\vspace{-1em}
The surface traction $\bm{t}(\bm{x},\hat{\bm{n}})$ acting on $\mathcal{B}$ is given by 
\begin{equation*}
\bm{t}(\bm{x},\hat{\bm{n}}) = -\rho_w g x_3 \hat{\bm{n}},
\end{equation*}
where $\hat{\bm{n}}$ is the outer unit normal to the surface $\partial \Omega_t$ and $\rho_w$ is the (constant) density of water and $g$ is the acceleration due to gravity. 

\medskip
(a) Determine the net force and moment acting on $\mathcal{B}$ via volume integrals.
Suppose the density varies linearly in the $x_2$ direction as
$$
\rho(x_2) = \rho_w \left( 1 + \frac{x_2}{2R} \right),
$$
so that at $x_2=-R$ we have $\rho = 0.5\rho_w$ and at $x_2=R$ we have $\rho=1.5\rho_w$.

The general net force is
$$
\bm F = g \,(\int_\Omega \rho(x_2)\,dV - \rho_w V)\,\bm e_3,
$$
Because the average value of $\rho(x_2)$ across the symmetric sphere equals $\rho_w$. Thus
$$
\bm F = \bm 0.
$$

The net moment is
$$
\bm M = g \int_\Omega \rho(x_2)\,\bm x\,dV \times \bm e_3,
$$
because the surface traction orients through the center of mass and contributes zero moment.

By symmetry, only the $x_2$-component survives:
$$
\int_\Omega \rho(x_2)\,x_2\,dV 
= \rho_w \int_\Omega \left( x_2 + \frac{x_2^2}{2R} \right) dV.
$$
The first term vanishes (odd in $x_2$), leaving
$$
\int_\Omega \rho(x_2)\,x_2\,dV 
= \frac{\rho_w}{2R}\int_\Omega x_2^2 \, dV.
$$

For a sphere of radius $R$,
$$
\int_\Omega x_2^2 \, dV 
= \frac{1}{3}\int_\Omega r^2 \, dV 
= \frac{1}{3}\cdot \frac{4\pi R^5}{5} 
= \frac{4\pi R^5}{15}.
$$

Therefore,
$$
\int_\Omega \rho(x_2)\,\bm x\,dV = \frac{2\pi}{15}\rho_w R^4\,\bm e_2,
$$

Finally, since $\bm e_2 \times \bm e_3 = \bm e_1$,
$$
\bm M = \tfrac{2}{15}\pi \rho_w g R^4 \,\bm e_1.
$$
\medskip
(b) Under what \textit{two} conditions is $\mathcal{B}$ in static equilibrium?

For a sphere with a linear density gradient along $x_2$, external loadings are needed to keep it in equilibrium:
$$
\bm F = 0, \qquad \bm M_{external} = -\tfrac{2}{15}\pi \rho_w g R^4 \,\bm e_1.
$$

\bigskip
\subsection*{2--3. \textbf{Viscoelastic data} [4 pts].} 
Stress relaxation \textcolor{red}{(Ask yourself: does it actually matter whether it's stress relaxation or creep compliance?)} isochrones for a compliant viscoelastic material are shown in the figure below.  

\begin{figure}[H]
\vspace{-1em}
\centering
\includegraphics[scale = 1.5]{instr-figures/PS2-Q3.pdf}
\caption{\small{Stress (Pa) vs. strain ($-$) for a soft viscoelastic material.}}
\end{figure}

\vspace{-1em}
(a) Are these isochrones from a material which we can describe with linear viscoelasticity? If not, why not, and if so, under what approximate regimes would this assumption be valid? 

For a linear viscoelastic (LVE) material, each isochrone $\sigma(\varepsilon,t)$ is a straight line through the origin with slope $E(t)$. 
In the figure, curves are nearly linear below $\sigma = 200 Pa$ but bend at larger stress and longer time. 
Thus, the LVE assumption is valid only in the small-strain, small-stress regime.

\medskip
(b) Estimate the creep relaxation function $J_c$ for stress values of 100 and 250 \textcolor{red}{Pa}. Isochrones are shown at times of 2, 5, 10, 20, and 40 seconds.   
% This is a placeholder for the example problems from the second problem set. 
% You'll replace this file with the one I supply on canvas. 

\noindent\textbf{At $\sigma_0=100~\mathrm{Pa}$:}
$$
\begin{array}{c|ccccc}
t~[\mathrm{s}] & 2 & 5 & 10 & 20 & 40\\ \hline
\varepsilon(t) & 4.25\times10^{-4} & 5.50\times10^{-4} & 6.25\times10^{-4} & 8.00\times10^{-4} & 1.00\times10^{-3}
\end{array}
$$
$$
J_c(t)\;=\;\frac{\varepsilon(t)}{\sigma_0}
\;=\;
\{\,4.25,\;5.50,\;6.25,\;8.00,\;10.0\,\}\times10^{-6}\ \mathrm{Pa}^{-1}.
$$

\noindent\textbf{At $\sigma_0=250~\mathrm{Pa}$:}
$$
\begin{array}{c|ccccc}
t~[\mathrm{s}] & 2 & 5 & 10 & 20 & 40\\ \hline
\varepsilon(t) & 1.20\times10^{-3} & 1.45\times10^{-3} & 1.80\times10^{-3} & 2.40\times10^{-3} & 3.00\times10^{-3}
\end{array}
$$
$$
J_c(t)\;=\;\frac{\varepsilon(t)}{\sigma_0}
\;=\;
\{\,4.80,\;5.80,\;7.20,\;9.60,\;12.0\,\}\times10^{-6}\ \mathrm{Pa}^{-1}.
$$
\bigskip
\subsection*{2--4. \textbf{Impulsive stresses} [4 pts].}

(a) Say that instead of a step load, we apply $\sigma(t) = A \delta(t)$ to an unknown linear viscoelastic material. 
Determine the strain history $\epsilon(t)$, first as a general function of the creep relaxation function $J_c(t)$, and then for a Kelvin-Voigt solid.

$$
\varepsilon(t)\;=\;\int_{0}^{t} J_c\!\big(t-\tau\big)\,d\sigma(\tau).
$$
Since $d\sigma(\tau)=A\,\delta'(\tau)\,d\tau$, we have
$$
\varepsilon(t)
=\int_{0}^{t} J_c(t-\tau)\,A\,\delta'(\tau)\,d\tau
= A\,\frac{d}{dt}J_c(t)
\quad\Rightarrow\quad
\,\varepsilon(t)=A\,\dot J_c(t)\,.
$$

\emph{Kelvin--Voigt:} $J_c(t)=\dfrac{1}{E}\big(1-e^{-t/\tau}\big),\ \tau=\dfrac{\eta}{E}$, so
$$
\dot J_c(t)=\frac{1}{\eta}\,e^{-t/\tau}
\quad\Rightarrow\quad
\,\varepsilon(t)=\dfrac{A}{\eta}\,e^{-t/\tau}\,.
$$
(b) Now, consider a rapid load followed by a rapid reverse load by applying a doublet function of stress, i.e. $\sigma(t) = B \psi(t)$. 
What is the strain function $\epsilon(t)$ in terms of $J_c(t)$ and for a Kelvin-Voigt material now? 

Now $d\sigma(\tau)=B\,\delta''(\tau)\,d\tau$, hence
$$
\varepsilon(t)
=\int_{0}^{t} J_c(t-\tau)\,B\,\delta''(\tau)\,d\tau
= B\,\frac{d^2}{dt^2}J_c(t)
\quad\Rightarrow\quad
\,\varepsilon(t)=B\,\ddot J_c(t)\,.
$$

\emph{Kelvin--Voigt:}
$$
\ddot J_c(t)=-\,\frac{1}{\eta\,\tau}\,e^{-t/\tau}
= -\,\frac{E}{\eta^{2}}\,e^{-t/\tau}
\quad\Rightarrow\quad
\,\varepsilon(t)=-\,\frac{B\,E}{\eta^{2}}\,e^{-t/\tau}\,.
$$
\bigskip
\subsection*{2--5. \textbf{Two-element models} [8 pts].}

Dynamic mechanical analysis (DMA) is a common technique for characterizing viscoelasticity. 
DMA conventionally involves application of a sinusoidal displacement to the top surface of a sample at a controllable temperature. 
Often, the user puts the sample into initial compression, and follows with the sinusoidal profile. 
A cylindrical sample of height $h$ and diameter $d$ is placed between two plates.
The DMA then quickly puts the sample into compression by moving its top plate downward by a displacement $d$, and then oscillates sinusoidally between positions $0$ and $2d$ at a frequency $\omega$.

\medskip
(a) Using the constitutive law for a Kelvin-Voigt material, determine the stress $\sigma(t)$ exerted by the platens to cause the applied strain. 
Applied strain is
$$
\varepsilon(t)=-\varepsilon_0-\varepsilon_0\sin(\omega t),\qquad \varepsilon_0=\frac{d}{h}.
$$
Stress:
$$
\sigma(t)=-E\varepsilon_0 - E\varepsilon_0\sin(\omega t) - \eta \varepsilon_0 \omega \cos(\omega t).
$$
\medskip
(b) The resulting stress lags behind the strain by an phase $\delta$, as in $\sin(\omega t + \delta)$. 
Commonly this is reported as the ``tangent loss'', or $\tan\delta$, for a material. 
What is the value of $\tan\delta$ for this particular Kelvin-Voigt model?

Combine harmonic terms:
$$
\sigma_{\text{ac}}(t)=\varepsilon_0\sqrt{E^2+(\eta\omega)^2}\,\sin(\omega t+\delta),
$$
$$
\tan\delta=\frac{\eta\omega}{E}.
$$
\medskip
(c) Say instead of prescribing the strain $\epsilon(t)$, we instead prescribed the stress, $\sigma(t) = - \sigma_0 - \sigma_0 \sin(\omega t)$. 
Determine the strain $\epsilon(t)$ for this prescribed stress.

$$
\varepsilon(t)=\int_{0}^{t} J_c(t-\tau)\,d\sigma(\tau).
$$
If $\sigma(t)$ is absolutely continuous, this is
$$
\varepsilon(t)=\sigma(0^+)J_c(t)+\int_{0}^{t} J_c(t-\tau)\,\dot\sigma(\tau)\,d\tau.
$$

For Kelvin--Voigt,
$$
J_c(t)=\frac{1}{E}\big(1-e^{-t/\tau}\big),\quad
\dot J_c(t)=\frac{1}{\eta}e^{-t/\tau},\quad \tau=\eta/E.
$$
So
$$
\varepsilon(t)=\int_{0}^{t}\frac{1}{\eta}e^{-(t-\tau)/\tau}\,\sigma(\tau)\,d\tau.
$$

With $\sigma(t)=-\sigma_0-\sigma_0\sin(\omega t)$,
the strain is
$$
\varepsilon(t)=-\frac{\sigma_0}{E}
+\frac{\sigma_0}{\sqrt{E^2+(\eta\omega)^2}}\;\sin(\omega t-\delta),
\qquad
\tan\delta=\frac{\eta\omega}{E}.
$$
\medskip
(d) Prove that the tangent loss function $\tan\delta$ is identical between the two loading methods.

It has been proved in (c)




