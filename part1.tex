\section*{Project I: Topic ID and Overview (due Sept 17)}

%This is just a placeholder for now

The first step in your semester-long research proposal development is to select a topic area in the mechanics of soft materials that's sufficiently interesting to you. 
It may be helpful to think of the proposal-writing process as the following. 
You want to study something that you are especially interested in, but you don't yet have the resources that you need to pursue this fully. 
Your job is (eventually) to communicate what it is you want to study, why it's worthwhile to be studied, and enumerate all of the reasons it's in some benefactor's interest to provide you the support that you need.
The particular benefactor we will leverage is the National Science Foundation, which cares about making fundamental ``vertical'' advances in fields (known as their \textit{Intellectual Merit} criterion) and having their funded projects improve society (known as the \textit{Broader Impact} to society criterion). 

Aim for approximately 500 words of total text, such as to reflect the important three Cs: \textbf{\textit{clear}}, \textbf{\textit{concise}}, and \textbf{\textit{compelling}}.
Your submission should be structured in three sections as separated below, and should address the following points: 

\begin{enumerate}
\item \textbf{Statement of Research Interest (why you personally want to study the subject)}
\begin{itemize}
\item Describe an area or phenomenon in the mechanics of soft materials that you find compelling. 
\item What motivates your interest and pursuit of this subject (e.g., your current or developing expertise, research interests, or otherwise)? \textit{Note: This may be more personal or anecdotal and is for my own understanding of your topic selection!}
\end{itemize}
\item \textbf{Intellectual Merit (why it is objectively worth delving deeper)}
\begin{itemize}
\item Describe, to someone with expertise in mechanics but perhaps not your system of interest, the core scientific principles underpinning (or perhaps, enabling development in) your topic of interest. 
\item Given the course syllabus, how will particular material we will cover this semester relate to what you propose? What background information do you need to do not just a good, but great, job in proposing something interesting? 
\end{itemize}
\item \textbf{Broader Impact (who, or what, does studying this area benefit?)}
\begin{itemize}
\item How might advances you envision in this area be impactful beyond your own interest? 
\item What does a ``winning scenario'' in this area look like? Briefly describe who might benefit (e.g., particular industries, health/science sectors, the public) and how that could plausibly happen.
\end{itemize}
\end{enumerate}

\emph{A strong submission will clearly illustrate your personal goals with this project, and show how your interest could manifest as advances in the broader field and society.}





