\section*{Project IV: Central Hypothesis and Data Analysis (Nov 17)}

%This is just a placeholder for now

Step four of your whitepaper development is the development of your central hypothesis along with your preliminary data, which underpins how you formulated that central hypothesis. 
This hypothesis should directly build off the ``promising lead'' you developed in the last project checkpoint. 

% Note that you are \textbf{not} expected or required to perform analysis at this stage---that will be the next project checkpoint. 
% Instead, focus on actually finding these datasets, understanding what they tell us, and briefly contextualizing these results in relation to both the original study and your own project. 

You'll execute this particular aim as another Powerpoint slide deck of a few slides, saved down as a pdf. \textit{A strong submission will contain all of the following:}


\renewcommand{\outlinei}{itemize}
\begin{enumerate}
\item[\textbf{1.}] \textbf{Central Hypothesis}
\begin{outline}
\1 Formulate a clear, precise, and testable hypothesis that is supported by your chosen analyzed data.  
\2 Note that this must be your best bet, out of all possibilities, as to the most likely outcome of your project's overall objective. Your central hypothesis is what gives direction to your research project. \textit{Note: if there's zero chance that your hypothesis is false, it's not a hypothesis! It's already proven!} 
\2 Your central hypothesis should be written at a relatively general level. You'll later develop specific aims for proving sequential aspects to what you write here. 
\2 \textit{Example: The central hypothesis of the proposed research is that incorporating mechanics information via theory and numerical approaches directly into the initial experimental setup will successful identification of spatially varying constitutive behavior of soft materials with internal gradients.}
\end{outline}
\item[\textbf{2.}] \textbf{General description of how you arrived at your hypothesis}
\begin{outline}
\1 Single sentence descriptions of the following:
\2 A reiteration of what the gap in the literature or knowledge base that you've identified for your proposed project.
\2 What in the data you found that connects importantly to your hypothesis.
\2 Briefly, the road map as to how you'd prove your hypothesis to be true given some resources and time.
\end{outline}
\item[\textbf{3--4.}] \textbf{Individual dataset analyses}
\begin{outline}
\1 For each of two selected datasets, include:
\2 A recap of that dataset with a concise description of that data (e.g. what trends, correlations, oddities are present) as compared to any control tests.
\2 Your own quantitative analysis of a specific figure/table/dataset in the context of your problem that somehow justifies that your hypothesis is a valid one to pursue. 
\2 A statement on the quantified limitations of the data in the context of your problem (e.g. trends are right, but we lack data in the frequency regime of interest). 
\2 Next steps/how to test this more thoroughly if you had more resources and time. Establish what would need to be done to confirm e.g., a trend, model, or assumption. 
\2 A link back to the central hypothesis here and knowledge gap in Project II. 
\end{outline}
\end{enumerate}








